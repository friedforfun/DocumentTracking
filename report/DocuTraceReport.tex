\documentclass[11pt]{article}
\usepackage[table]{xcolor}
\usepackage{geometry}
 \geometry{
 a4paper,
 total={170mm,257mm},
 left=20mm,
 top=20mm,
 }
\usepackage{amsmath}
\usepackage{graphicx}

\usepackage{setspace}

\usepackage{amssymb}
\usepackage{epstopdf}
\usepackage{inputenc}

\usepackage{dashrule}
\usepackage{float}
\usepackage{hyperref}
\usepackage{url}
\usepackage{mwe}
\usepackage{caption}
\usepackage{longtable}
\usepackage{minted}
\usepackage{xr}


\usepackage[backend=biber,style=ieee]{biblatex}
\usepackage[toc]{appendix}
\usepackage[acronym]{glossaries}

\addbibresource{IndustrialProg.bib}

\hypersetup{ linktoc=all}
\graphicspath{ {./images/} }


\begin{document}
\title{%
	\bf DocuTrace\\ 
    \large F20FC: Industrial Programming\\
    Coursework 2}

\author{
	Sam Fay-Hunt | \texttt{sf52@hw.ac.uk}
}

\maketitle
\thispagestyle{empty}
\pagebreak


\tableofcontents
\thispagestyle{empty}
\pagebreak


\setcounter{page}{1}

\emph{The report should have between 10–15 pages and use the following format (if you need space for additionalscreenshots, put them into an appendix, not counting against the page limit, but don’t rely on the screenshotsin your discussion)}

\section{Introduction}
\emph{State the purpose of the report, your remit and any assumptions you have made duringthe development process.}
The DocuTrace application is a moderate size, data-intensive application, its purpose is to analyse and display document tracking data from the website \href{https://issuu.com/}{issuu.com}. 
The website hosts a substantial number of documents, and provides anonymised usage statistics, the data is provided in the form of a sequence of individual JSON entries seperated by new lines.
It is assumed the users of an application like DocuTrace would be someone with enough degree of technical competency to use simple Linux command line applications, the user would likely be a researcher (data science), or a business. 
The prior assumption leads to the assumption that the hardware running this application would be closer to server class than standard consumer hardware with higher CPU core counts and alot more RAM.
Due to the potential scale of the data a  signifantly large amount of RAM is not mandatory to run this application, but a pool of aproximately 8GB of RAM should be installed on the system when processing 3 million lines otherwise a significant performance penalty may be incurred.
DocuTrace was written in Python 3, it is intended to be run on Ubuntu 20.04, and has not been tested on other operating systems.~\href{https://www2.macs.hw.ac.uk/~sf52/DocuTrace/html/readme.html}{Comprehensive documentation} has been generated, a list of dependencies, installation and run instructions are provided, note the recommended entryjoint of the application is via the ``docutrace'' shell script, this script will automatically configure an environment variable to specify the output directory of graph files generated.


\section{Requirements Checklist}
\emph{Here you should clearly show which requirements you have delivered andwhich you haven’t.}

\section{Design Considerations}
\emph{Here you should clearly state what you have done to your application tomake it more usable and accessible.}

\section{User Guide}
\emph{Use screen shots of the running application along with text descriptions to help youdescribe how to operate the application.}

\section{Developer Guide}
\emph{Describe your application design and main areas of code in order to help another developer understand your work and how they might develop it. You may find it useful to supplementthe text with code fragments.}

\section{Testing}
\emph{Show  the  results  for  testing  all  cases  and  prove  that  the  outputs  are  what  are  expected.Preferably,  use  unit  testing  to  test  core  functionality  of  the  implementation.   If  certain  conditionscause erroneous results or the application to crash then report these honestly.}

\section{Personal Development}
\emph{A short discussion on lessons learnt from the feedback given on CW1and a discussion how you integrated this feedback into CW2.  Cover both coding and report writing,possibly more (project management, preparing for interview style questions etc).}
Lessons learnt from the experience of CW1:
Started out using test driven development


Feedback from CW1:
 --- code ---
Less code duplication
too much global state
limited input validation
no custom exceptions
not restrictive enough access modifiers

--- report ---
intro:
    should cover short spec
    cover goals
    cover env

dev section:
    should discuss class dependencies
    should discuss method interfaces := params/return, assumptions of args

conclusion:
    should discuss adv lang features




\section{Conclusions}
\emph{Reflect on what you are most proud of in the application and what you’d have likedto have done differently.  You should reflect on the produced software, and compare software devel-opment in scripting vs.  systems languages.}
Most proud of:
- Concurrency during file reading
- Structure of the code, good decoupling front and back end

Do differently:
- Leave more time to work on the gui
- Use a different library for the gui
- subclass the datacollector class to break it into smaller tasks


\pagebreak
\appendix
\section{References}
\printbibliography

\end{document}
